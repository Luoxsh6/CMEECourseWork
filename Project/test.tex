\documentclass[11pt]{article}

\usepackage[left=2cm,right=2cm,top=2cm,bottom=2cm]
{geometry}
\usepackage{lineno}
\usepackage{setspace}
\usepackage[round]{natbib}

\title{\textbf{Scaling, metabolic theory and animal activity budgets}}
\author{\textbf{Richard Cornford} 
		\\
		\\
		\\
		Supervisors:
		\\
		\\
		Samraat Pawar, Department of Life Sciences, Imperial College London
		\\
		\\
		Chris Carbone, Institute of Zoology, Zoological Society of London}

\begin{document}

\maketitle

\pagebreak 
\linenumbers
\onehalfspacing

\section*{Introduction}
It is well documented that body mass greatly influences many aspects of animal physiology and anatomy due to allometric scaling relationships \citep{randall2002eckert, schmidt1984scaling, peters1986ecological}. Empirical data has demonstrated that metabolic rate, locomotory costs and consumption rates all scale with body mass in the form of power functions \citep{schmidt1984scaling, kleiber1932body, taylor1982energetics, shipley1994scaling, pawar2012dimensionality}. In particular, because mass specific metabolic rates decline with increasing size (\begin{math}mass^{-0.25}\end{math}) it has long been predicted that the proportion of time an animal spends active should also decline in an associated manner, assuming that consumption rates scale in an equivalent fashion \citep{peters1986ecological, rizzutoinpress}. However, recent work has shown that in terrestrial carnivores a single scaling relationship between mass and activity time does not occur. Instead, for small hunters time active actually increase with increasing body size whilst in larger hunters the expected negative exponent is observed \citep{rizzutoinpress}. Interestingly, a mechanistic model based on metabolic theory was able to predict this reversed scaling by incorporating ecological and biomechanical constraints known to influence small predators \citep{rizzutoinpress}. Through modification of the model components it is possible to predict activity budgets for animals with a range of dietary and locomotory types \citep{rizzutoinpress, pawar2012dimensionality}. Furthermore, both theory and observation also indicate that consumption rates scale differently depending on the dimensionality of the resource \citep{pawar2012dimensionality}. With more data on animal activity across a variety of habitat types, I hope to test the broader application of the model framework to a range of different ecological scenarios.
\\
\\
Depending on the quality/quantity of tracking data obtained there are a number of potential hypotheses I would like to address:
\\
Is the positive scaling of foraging time observed in small terrestrial carnivores also present when considering terrestrial herbivores or do herbivores follow the typical negative slope?
\\
Are activity patterns distinctly different between 2D (eg terrestrial) and 3D (eg pelagic) environments for a given feeding type?

\section*{Methods}
To complete this project I will utilise models developed in \citet{pawar2012dimensionality} and \citet{rizzutoinpress} and adapt them as required for the particular ecological context being considered.
\\
\\
In order to determine the accuracy of the models I aim to compile a large dataset of animal tracking data, including the downloading of data from the online repository Movebank. From this data I hope to be able to calculate the proportion of time an animal spends moving and in turn approximate its activity budget. I will therefore be able to compare, through statistical analysis, the empirical data to the predictions made by mechanistic models based on metabolic theory. 


\section*{Expected outcomes}
From this project I hope improve the understanding of the way in which animal behaviour is influenced by physiological and biomechanical constraints, potentially identifying differences resulting from the specific nature of the environments in which organisms live. 

\section*{Feasibility}
Project timeline, insert gant chart, 
\\
\\
General overview:
\\
First month or two - Reading, coding for data download, developing models.
\\
Month two onwards - Model fitting to observed data, comparison of theory and reality.
\\
Month three onwards - Writing report.

\section*{Budget}
I do not believe that this project will require a significant budget as it is desk/computer based.
\\
\\
However some money for covering printing costs and for travel related to furthering my knowledge of the topic area may be needed. 

\section*{Suggested markers}
At present I am unsure as to who exactly would be suitable but anyone with a background in modelling and/or behavioural ecology would probably be ideal. 

\newpage
\bibliographystyle{plainnat}
\bibliography{rc1015_proposal}
\end{document}
