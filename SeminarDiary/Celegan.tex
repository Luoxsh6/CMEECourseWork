\documentclass[11pt]{article}



\usepackage[left=4.1cm,right=4.1cm,top=2.97cm,bottom=5cm]%
{geometry}
\usepackage{graphicx}
\usepackage{gensymb}
\usepackage{amsmath}

\usepackage{setspace}
\usepackage{diagbox}



% Harvard-style referencing
\usepackage[comma]{natbib}

\setcounter{page}{12}

\begin{document}
\bibliographystyle{agsm}
\setcitestyle{authoryear,open={(},close={)}}


\newpage
\begin{document}\large
\font\myfont=cmr12 at 14pt 
\title{\myfont C.elegan as a tractable host to study natural infections by oomycetes}
\date{}        
\author{Dr. Michalis Barkoulas}
\maketitle
\noindent
C.elegan has been extensively used as a model organism in molecular and developmental biology, because of the benefits like short life cycle, easy to cultivate, transparent, simple anatomy and genome, etc. Oomycetes are eukaryotic organisms that inhabit a variety of terrestrial and aquatic enviroments and infect a range of animals and plants. Animals infections by oomycetes have been very litte studied due to paucity of tractable host, so Michalis's lab developed a new pathosystems based on the discovery of natural oomycete infections of C.elegans.
\\
\\
Michalis's lab reported a new natural oomycetes pathogen of C.elegans, M.humicola, which showed some host-specificity infections strategy: budding, hypha growth, sporangia and zoospore release by tractable label FISH (fluorescent in situ hybridisation). Moreover, a novel pathogen-specific immune responses was identified by studying the changes in gene expression with and without exposure to M.humicola infection, which was a previously uncharacterized gene family encoding chitinase-like (CHIL) proteins. They demonstrated that the response is higly specific against M.humicola and antagonizes the infections. 
\\
\\
Further, they found that animals overexpressing chil genes showed reduced pathogen attachment, and vice versa. They propose that chil proteins may diminish the ability of the oomycete to infect by hindering pathogen attachment to the host cuticle. By using atomic force microscopy (AFM) to generate force displacement curves whereby the displacement of the cuticle is measured upon delivering quantifiable forces and they found that chil gene overexpression significantly changed the stiffness of cuticle.
\\
\\
Overall, their research developed a new pathosystems to better understand how C.elegans sense and respond to oomycetes.
\end{document}
