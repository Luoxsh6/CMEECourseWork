\documentclass[11pt]{article}



\usepackage[left=4.1cm,right=4.1cm,top=2.97cm,bottom=5cm]%
{geometry}
\usepackage{graphicx}
\usepackage{gensymb}
\usepackage{amsmath}

\usepackage{setspace}
\usepackage{diagbox}



% Harvard-style referencing
\usepackage[comma]{natbib}

\setcounter{page}{6}

\begin{document}
\bibliographystyle{agsm}
\setcitestyle{authoryear,open={(},close={)}}


\newpage
\begin{document}\large
\font\myfont=cmr12 at 14pt 
\title{\myfont Parallel adaptation of rabbit population to myxoma virus}
\date{}        
\author{Prof. Frank Jiggins}
\maketitle
\noindent
The myxoma virus was released into wild rabbits in Australia and Europe in the 1950s, while the rabbits in both continents rapidly evolved resistance to the virus. Frank's team investigated the genetic basis of this resistance by comparing the exomes of modern individuals with the historical rabbit specimens collected before the virus release and found out a strong pattern of parallel evolution, with selection on standing genetic variation favoring the same alleles in Australia, France, and the United Kingdom and many of these changes occurred in immunity-related genes.
\\
\\
By sequencing a total of 152 rabbits from Australia, France, and the United Kingdom. Genome-wise polymorphism data analysis showed the colonization route of rabbits is from French to the United Kingdom and then Australia. Genetic variation in historical and modern populations showed that similar patterns of genetic structure and diversity from the same country, more generally, across all SNPs, the allele frequencies of historical and modern population are highly correlated in the three countries. In order to investigate the parallel genetic changes occurred across three countries, they calculated the fixation index between the historical and modern samples for each country and identifide the 1000 SNPs that show the highest fixation index. Moveover, among these SNPs, differentiated in any two populations tend to show elevated fixation in the third population. Desipite the common selection pressure imposed by myxomatosis, they also experienced their distinct selection pressures quantified by Bayesian approach, perhaps due to differences in ecology.
\\
\\
Further investigation into the roles of SNPs subjected to selection revealed two strategies to evolve resistance: selection on the immune system, to increase the potency of the IFN response, reduced the replication of an attenuated strain of MYXV, and changes in host pro-viral proteins that viruses hijack for their own benefit, evaluation of the role of VPS4 found strongly inhibitation in MYXV replication.



\end{document}
