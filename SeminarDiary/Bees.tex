\documentclass[11pt]{article}



\usepackage[left=4.1cm,right=4.1cm,top=2.97cm,bottom=5cm]%
{geometry}
\usepackage{graphicx}
\usepackage{gensymb}
\usepackage{amsmath}

\usepackage{setspace}
\usepackage{diagbox}



% Harvard-style referencing
\usepackage[comma]{natbib}

\setcounter{page}{9}

\begin{document}
\bibliographystyle{agsm}
\setcitestyle{authoryear,open={(},close={)}}


\newpage
\begin{document}\large
\font\myfont=cmr12 at 14pt 
\title{\myfont The microbial ecology of bees, and engineering protective microbiomes}
\date{}        
\author{Dr. Peter Graystock}
\maketitle
\noindent
Pollinators play a critical role in ecosystem functions and services. Recently, researchers have reported pollinator diversity and abundance decline, especially honeybees and bumblebees in Europe and North America. This decrease may be caused by pathogens, parasites, insecticides, pollutants, habitat fragmentation and climate change. A survey on the prevalence of 5 parasites in 3000 bees and 3000 flowers were carried out in Peter’s previous research, he detected that 77\% commercially imported bumblebee colonies carried several infectious parasites. This may impose detrimental impacts on other indigenous pollinators since the transmission can be facilitated by flowers. Using an experimental approach, Peter indeed proved that flowers can be a hub for parasite transmission.
\\
\\
In Peter’s talk, he pointed out the significance of gut bacteria of bees. First, the gut bacteria communities vary among host species and is linked to host sociality. Second, gut microbiomes can help reduce parasite virulence. This antagonistic relationship between gut bacteria community and parasite (Crithidia bombi) has been supported by data both from lab and field. However, it is still unclear why gut bacteria can function as a protection of parasite virulence. There may be competition of limited resources when parasites co-exist with gut bacteria. Another possibility is direct interactions between gut bacteria and parasites and leads to parasite fitness decreases. Due to the flaw of experimental design here, researchers can not exclude the possibility that the antibiotic treatment (i.e. host without gut bacteria) led to higher mortality because of lethal toxin effect. Last, bumblebee biome seems to help increase bumblebee tolerance to selenium pollutant in the soil. This ongoing research is trying to address the role microbiome plays in bumblebee survival under selenium threat by testing the metabolic responses in different microbiome combinations. Future work including metatranscriptomics is required to provide a comprehensive understanding of
bumblebee microbiome. 

\end{document}
