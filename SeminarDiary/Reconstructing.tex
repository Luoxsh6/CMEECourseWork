\documentclass[11pt]{article}



\usepackage[left=4.1cm,right=4.1cm,top=2.97cm,bottom=5cm]%
{geometry}
\usepackage{graphicx}
\usepackage{gensymb}
\usepackage{amsmath}

\usepackage{setspace}
\usepackage{diagbox}



% Harvard-style referencing
\usepackage[comma]{natbib}

\setcounter{page}{7}

\begin{document}
\bibliographystyle{agsm}
\setcitestyle{authoryear,open={(},close={)}}


\newpage
\begin{document}\large
\font\myfont=cmr12 at 14pt 
\title{\myfont Reconstructing hyperdiverse food webs: fish gut content metabarcoding as a tool to disentangle trophic interactions on coral reefs}
\date{}        
\author{Dr. Jordan Casey}
\maketitle

\noindent
Understanding the role of predators in food webs can be challenging in complicated systems such as coral reefs which composed of small crptic species. DNA based dietary analysis can provide supplement predator removal experiments and provide a high-resolution and large-scale trophic webs of coral fishes in Moorea, French Polynesia (17\degree30'S, 149\degree50'W). The study is benefits from Moorea BIOCODE project and the construction of cytochrome c subunit I (COI) sequence library. The Moorea BIOCODE project aims at establishing molecular markers for all non-microbial species of French Polynesia tropical ecosystem and it has identified over 5670 macrobiotic species and constructed species-specific DNA barcode library for most animals. 
\\
\\
Identifying species from these samples relies on the ability to match sequences with reference barcodes for taxonomic identification, while ribosomal markers are targeted in most of the previous studies, despite the fact that the mitochondrial Cytochrome c Oxidase subunit I gene(COI) is by far the most widely available sequence region in public reference libraries. Largely because the available universal COI primers target the 658 barcoding region, which is consider to large for sequencing. So, Jordan's team had design a new PCR primer within the higly variable mitochondrial COI region, the "mlCOIintF". 
\\
\\
With these primer for species identification in fish gut content, she showed that most of fish species spread out and some species showed high degree dietary hybridization. Moreover, there are limited overlaps among fish family but the NMDS result still showed a cluster pattern. However, only family significantly explained diet partitions while no significant effect of trophic category. Overall, this study revealed a detailed and broad view of food web trophic coral reef system. 

\end{document}
