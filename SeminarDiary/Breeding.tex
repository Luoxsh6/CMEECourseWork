\documentclass[11pt]{article}



\usepackage[left=4.1cm,right=4.1cm,top=2.97cm,bottom=5cm]%
{geometry}
\usepackage{graphicx}
\usepackage{gensymb}
\usepackage{amsmath}

\usepackage{setspace}
\usepackage{diagbox}



% Harvard-style referencing
\usepackage[comma]{natbib}


\setcounter{page}{5}
\begin{document}
\bibliographystyle{agsm}
\setcitestyle{authoryear,open={(},close={)}}


\newpage
\begin{document}\large
\font\myfont=cmr12 at 14pt 
\title{\myfont Breeding system evolution in light of demographic sex biases}
\date{}        
\author{Dr. Luke Eberhart-Phillips}
\maketitle
\noindent
Sex ratio variation is an important characteristics of the demographic and evolution study. In particular, adult sex ratio (ASR) shows remarkable variation throughout nature, with birds and mammals tending to have male-biased and female-biased ASRs. In Luke's study, he mainly focused on the demographic pathways that shape sex biase and assess their evolutionary consequences on the parental and mating strategies across several species of \textit{Charadrius} plovers.
\\
\\
Firtly, Luke tried to answer the question that at which point in the life time do these biases emerge? And He reported a detailed individual-based demographic analysis of an intensively studied wild bird population to evaluate origins of sex biases and their consequences on mating strategies and population dynamics. He found that in a polygamous snowy plovers, the primary driver of male-baised sex ratio is sex-specific juvenile survival rather than adults or dependent offspring. And the finding provides support for theories of unbiased sex allocation when sex differences in survival arise after the period of parental investment.  Sex biases also strongly influenced population viability, which was significantly overestimated when sex ratio and mating system were ignored. Then he demostrated the significance of sex differences of juvenile survival in determining ASR bias using 6119 individuals from six wild shorebird pupulations.
\\
\\
So it is conceivable that juvenile survival contributes mostly in biased ASR since it is a critical stage in speices development when they are exposed to complicated environment with few survival skills. As for how these biases arises in juvenile survival? We can probably find some intrinsic factors such as sex-diffrent ontogeny and genotype-sex interactions.

\end{document}


