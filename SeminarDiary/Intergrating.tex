\documentclass[11pt]{article}



\usepackage[left=4.1cm,right=4.1cm,top=2.97cm,bottom=5cm]%
{geometry}
\usepackage{graphicx}
\usepackage{gensymb}
\usepackage{amsmath}

\usepackage{setspace}
\usepackage{diagbox}



% Harvard-style referencing
\usepackage[comma]{natbib}


\setcounter{page}{8}
\begin{document}
\bibliographystyle{agsm}
\setcitestyle{authoryear,open={(},close={)}}


\newpage
\begin{document}\large
\font\myfont=cmr12 at 14pt 
\title{\myfont Integrating ecology across scales of biological organisation}
\date{}        
\author{Dr. Diego Barneche}
\maketitle
\noindent
The main topic of Diego's talk is to discuss how individual-level determinants of metabolism as well as life-history traits affect the energy flux and overall productivity of populations, communities and ecosystems. 
\\
\\
The metabolic theory of ecology (MTE) posits that the metabolic rate of organisms is the fundamental biological rate that governs most observed patterns in ecology. It is based on an interpretation of the relationships between body size, body temperature, and metabolic rate across all organisms. To explain the relationship between body mass and temperature, the most known equation is:\begin{equation*} B=b_0M^{\frac 3 4}e^{-\frac E {kT}} \end{equation*} 
where $b_0$ is a mass-independent normalization constant, M is organism mass, E is activation energy in electronvolts or joules, T is absolute temperature in kelvins ,and k is the Boltzmann constant.
\\
\\
Diego showed a series of studies metabolic framework. The results of relationship between matebolic rate and body mass, and the temperature-dependance energy flux empirically follows the MTE. Meanwhile, he demonstrate that the cost of growth, $E_m$ varies substantially among fishes, and that it may increase with temperature, trophic level and level of activity. Moreover, the system carrying capacity will drop significantly when suffered from extremely heat.
\\
\\
Another important proposition of metabolic theory is how the body size determines the reproductive-energy output. He showed that larger mothers reproduce disproportionately more than smaller mothers contribute disproportionately in fecundity and total reproductive energy, which reset the theory on how reproductive scales with size. 

\end{document}
