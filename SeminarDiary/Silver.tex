\documentclass[11pt]{article}



\usepackage[left=4.1cm,right=4.1cm,top=2.97cm,bottom=5cm]%
{geometry}
\usepackage{graphicx}
\usepackage{gensymb}
\usepackage{amsmath}

\usepackage{setspace}
\usepackage{diagbox}



% Harvard-style referencing
\usepackage[comma]{natbib}

\setcounter{page}{11}

\begin{document}
\bibliographystyle{agsm}
\setcitestyle{authoryear,open={(},close={)}}


\newpage
\begin{document}\large
\font\myfont=cmr12 at 14pt 
\title{\myfont The Evolution of Silver Spoon Effects}
\date{}        
\author{Dr. Barbara Tschirren}
\maketitle
\noindent
The life-long reproductive advatange (i.e. increased fitness) enjoyed by an individual that had access to abundant resources during the early. In Barbara's talk, she explained why not all parents provide favorable conditions for their offspring, how parental provisioning experienced by an individual during development affects its own offspring provisioning later in life, and how such cascading parental effects shape the evolution of parental care.
\\
\\
As well as passing on genes, a mother shapes her offsprings phenotype by influencing the environment they experience early in life.However, whether the traits causing these marernal effects also affect their own expression in subsequent generations (cascading maternal effects) and the evolution implicationss of such feedback loops are not well understood. In Barbara's research, they reported that the investment a mother makes in her eggs positively affects the egg investment of her daughter, the size of eggs daughters lay resembles the egg size of their maternal line significantly more than that of their paternal line in Japanese quail, highlighting that egg size is in part maternally inherited. Furthermore, this maternal variance in offspring egg size can be explained by maternal egg size, which can support the presence of a positve cascading effect of maternal egg size on offspring egg size.
\\
\\
By using evolutionary modeling, she further demonstrated that this association between additive genetic and positive cascading maternal effects leads to anamplification effect, accelerating the evolutionary potential of both maternal investment and any other traits in offspring (e.g.,body size) affected by this maternal investment.
\\
\\
Overall, their study provides empirical evidence for positive cascading maternal effects, which by amplifying the amount of variation avaliable for selection to act on, affect the evolutionary potential of both prenatal maternal investment and juvenile body size. Evolutionary models showed that such positive cascading maternal effects only influence evolutionary dynamics in the presence of additive genetic effects. 
\end{document}
