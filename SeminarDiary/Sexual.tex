\documentclass[11pt]{article}



\usepackage[left=4.1cm,right=4.1cm,top=2.97cm,bottom=5cm]%
{geometry}
\usepackage{graphicx}
\usepackage{gensymb}
\usepackage{amsmath}

\usepackage{setspace}
\usepackage{diagbox}



% Harvard-style referencing
\usepackage[comma]{natbib}

\setcounter{page}{4}

\begin{document}
\bibliographystyle{agsm}
\setcitestyle{authoryear,open={(},close={)}}


\newpage
\begin{document}\large
\font\myfont=cmr12 at 14pt 
\title{\myfont Routes to the resolution of sexual conflict}
\date{}        
\author{Dr. Alison Wright}
\maketitle
\noindent
Sexual conflict occurs when males and females have different optimal fitness strategies concerning reproduction, leading to an evolutionary arms race between them. In Alison's presentation, she mainly discussed two topics: i. Does sexual conflict drive sex chromosome formation? ii. Can sexual conflict be resolved by regulatory evolution?
\\
\\
For the first question, firstly, they support the theory that sex chromosome evolve from autosomes, initially with the acquisistion of a sex determining locus. Then, the emergence of sexually antagonistic allele at loci in close proximity to the sex determining locus selects for recombination suppression between sex chromosome, once the recombination is halted on the sex chromosome, the non-recombination region can expand with the acquisistion of additional sexually antagonistic alleles and further recombination suppression. The lack of recombination leads to accumulation of repetitive DNA, which can lead to short-term increase in size but results in large-scale deletion in sex chromosome. They used guppies' colors which are inherited consistent wih Y-linkage as their research object, characterized their sex chromosome and find out how much recombination suppression between them by sequencing.
\\
\\
However, sex chromosome only represents a samll portion of the genome, if the idea about sexual conflict is true, then essentially the locus side with sexually antagonistic alleles should be distributed across whole genome. So they recently tested the realtionship between regulatory evolution and sexual conflict across autosomes and they showed that sex-biased expression in general, and perhaps male-biased expression in particular, is a rapid and effective route to resolve intra-locus sexual conflict.

\end{document}
