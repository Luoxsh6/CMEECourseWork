\documentclass[11pt]{article}



\usepackage[left=4.1cm,right=4.1cm,top=2.97cm,bottom=5cm]%
{geometry}
\usepackage{graphicx}
\usepackage{gensymb}
\usepackage{amsmath}

\usepackage{setspace}
\usepackage{diagbox}



% Harvard-style referencing
\usepackage[comma]{natbib}

\setcounter{page}{3}

\begin{document}
\bibliographystyle{agsm}
\setcitestyle{authoryear,open={(},close={)}}


\newpage
\begin{document}\large
\font\myfont=cmr12 at 14pt 
\title{\myfont Understanding the effectiveness of indoor based malaria control}
\date{}        
\author{Dr. Tom Churcher}
\maketitle
\noindent
Indoor residual spraying (IRS) and long-lasting insecticide treated nets (LLINS) are together contributed to the success of reducing malaria since 2000. However, the substantial reduction in global malaria rebounded in 2016, mainly due to increasing mosquito resistance to pyrethroid insecticides, which are the widely-used class of insecticide for LLINs and IRS. A potential solution is the next-generation bednets and IRS products, piperonyl butoxide (PBO), a syergist that can inhibits specific metabolic enzymes that can detoxify pyrethroid. What's more, several IRS products have recently become available. Novel products need not only show epidemiological impact but also should be recommended based on safety, quality and entomological efficacy data. Hence, Tom's talk outline how to combine entomological data with mathematical models to understand the optimum set of interventions needed to control malaria in a setting.
\\
\\
In Tom's talk, they systematically characterise different IRS product efficacies against mosquito by experimental hut data. Lab trials, hut trials and village trials all demonstrated the pyrethroid-PBO bednets is superior than the traditional pyrethroid bednets. Besides, he investigated the mosquitos' diel behaviour and host diel behaviour and quantify the timing and geographic pattern of mosquito biting activity, and demonstrate the association between host activity and mosquito activity. Overall, they provided a potential effective solution to control malaria with increasing insecticide resistance, though their benefit depends on local factors inculding bednet use, seasonality, endemicity and pyrethroid resistance statusof local mosquito populations.
\\
\\
The screening of formulations does require a lot of trails, Tom's research is a good combination of experimental trials data, statistical analysis and model fitting, truely provided a good standard to help decision makers evaluate IRS product effectivens.

\end{document}
