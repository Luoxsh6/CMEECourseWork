\documentclass[11pt]{article}



\usepackage[left=4.1cm,right=4.1cm,top=2.97cm,bottom=5cm]%
{geometry}
\usepackage{graphicx}
\usepackage{gensymb}
\usepackage{amsmath}

\usepackage{setspace}
\usepackage{diagbox}



% Harvard-style referencing
\usepackage[comma]{natbib}

\setcounter{page}{10}

\begin{document}
\bibliographystyle{agsm}
\setcitestyle{authoryear,open={(},close={)}}


\newpage
\begin{document}\large
\font\myfont=cmr12 at 14pt 
\title{\myfont Coexistance, community assembly and the N-dimensional hypervolume}
\date{}        
\author{Dr. Alex Pigot}
\maketitle
\noindent
'Why are there so many kinds of animals?', this question posed by G.E. Hutchinson remains one of the great challenges in ecology. In Alex's talk, he disscuss their work integrating phylogenies, functional traits and species geographic distributions for the global avian radiation, to understand how ecological niche limits sympatry in birds with a broad-scale and explored how many trait dimensions can describe a avian niche space.
\\
\\
Hutchinson' niche theory was described as a space consisted of both biotic and abiotic variables limiting the species viability. Species morphological traits can be critical variables that constrain species distribution and sympatric pattern. With the availability of high-resolution avian phylogeny and the establish of global avian morphology database, Alex is able to discuss the sympatry pattern in the entire avian clade under a global scale. 
\\
\\
Alex and his team revealed that body size is the only strong predictor when explaining coexistence among sister avian species in global scale. But body mass performs poor when it comes to prediction of trophic and foraging niches at larger taxonomic scales. Alex also demonstrated that avian niches are multi-dimensional instead of single linear. These results provide a novel perspective of avian coexistence and community assembly in a broad scale, implying morphological traits such as body size (including body mass and body length) and beak size are of great importance in avian multidimensional niches. This work filled the gaps of previous work focusing on regional or certain clade and suggested that avian niche space have more dimensions compared to plants.
\\
\\
In his recent work, he tired to use random forest to build a classifier for avian foraging niches and achieve around 80\% accuracy. But I am kind of confused about his theory about using PCA as a dimensions tool to evaluate the tarit dimensions, because PCA will take all tarits into account for dimensional reduction.
\end{document}
